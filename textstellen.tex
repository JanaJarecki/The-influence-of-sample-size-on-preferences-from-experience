\subsection{How does sample size impact preferences?}

Internal biases occur when people do not weight each sample equally or use only a subset of the observed sample to form their preference from experience. The most prominent internal bias discussed in the literature is the recency effect. The recency effect describes that samples observed later in a sequence predict choices better than earlier observations \citep[e.g.,][]{Hertwig2004,Rakow2008}. Some studies indeed found evidence of recency effects in decisions from experience \citep[e.g.,][]{Hertwig2004,Rakow2008}. Others, however, did not \citep[e.g.,][]{Hau2008,Ungemach2009} or even showed evidence for primacy effects instead \citep[e.g.,][]{Camilleri2011}. 


Descriptive and experience-based paradigms differ in how people learn about the properties of gambles. In descriptive paradigms, gambles are illustrated with a numerical, graphical, or text-based description of their outcome distributions. In experience-based paradigms, participants draw samples from unknown gambles to explore the gambles' outcome distributions. The more participants sample, the closer the observed samples will reflect the real outcome distribution. 
The literature on choice behavior suggests a systematic difference between choices from description and choices from experience, the so-called description--experience gap \citep[D--E gap;][]{Hertwig2004}. However, it is unclear how valuations depend on whether probabilities are presented descriptively or learned from experience. Therefore, we investigate in this study how people form valuations from experience and how valuations from experience differ from valuations from description. 



Second, we aim to gain a new perspective on the D--E--gap by studying valuations instead of decisions from description vs. experience. The D--E--gap holds that when people make decisions between gambles from description, they choose as if they overweight rare events \citep{Kahneman1979}. At the same time, when they make similar choices based on experience, they choose as if they underweight rare events \citep{Hertwig2004}. After the D--E gap was first described, critics argued that it solely results from sampling error, an external bias \citep{Hadar2009}

Such external bias occurs when a decision maker samples a sequence of outcomes that does not represent the gamble's true outcome distribution. This bias is most likely to occur if people draw small samples of outcomes  \citep[e.g.,][]{Hadar2009}. In a free-sampling paradigm, participants can in principle sample many outcomes to learn about the true outcome distribution. However, participants typically draw small samples. \cite{Hertwig2004} reported that participants drew a median of 15 observation per decision. With such a low number, the skew of a gamble's binomial sampling distribution implies that more people undersample than oversample the rare event. Averaged across trials, in Hertwig's 2004 study, the rare event was undersampled by $78\%$ of the participants. Other studies show similar sampling error \citep[e.g.,][]{Hau2008,Rakow2008}. Given this undersampling, it appears obvious that people behave as if they underweight rare events when deciding from experience.

Therefore, researchers have tried to reduce or eliminate sampling error to allow for an unbiased comparison between decisions from description and decisions from experience. Attempts have involved: Encouraging participants to draw larger samples by increasing the incentives \citep{Hau2008}, showing fixed samples that represent the gambles' objective probability distributions \citep{Ungemach2009}, showing fixed -- relatively large -- numbers of random samples as larger total samples naturally reduce sampling error \citep{Hau2008,Hau2010}, presenting gamble descriptions that match the outcome distribution that a ``partner'' participant had sampled \citep{Rakow2008}, and constraining statistical analysis to trials in which the total sample closely represented the gambles' underlying outcome-distribution \citep[e.g.,][]{Camilleri2009, Camilleri2011a}. Studies in which sampling error was reduced reported a diminished D--E gap \citep{Hau2008,Hau2010} or failed to find a D--E gap \citep{Gloeckner2012}. Studies in which sampling error was eliminated or accounted for by the statistical analysis also found a diminished D--E gap \citep{Ungemach2009} or no D--E gap at all \citep{Camilleri2009,Camilleri2011a,Rakow2008}. 

In summary, the D--E gap appears to be at least partly driven by external bias, i.e., sampling error. However, some studies reported a D--E gap in the absence of sampling error \citep{Hau2008, Hau2010, Ungemach2009}. This suggests that other, internal, biases also play a role when people form preferences from experience. 

Internal biases occur when people do not weight each sample equally or use only a subset of the observed sample to form their preference from experience. The most prominent internal bias discussed in the literature is the recency effect. The recency effect describes that samples observed later in a sequence predict choices better than earlier observations \citep[e.g.,][]{Hertwig2004,Rakow2008}. Some studies indeed found evidence of recency effects in decisions from experience \citep[e.g.,][]{Hertwig2004,Rakow2008}. Others, however, did not \citep[e.g.,][]{Hau2008,Ungemach2009} or even showed evidence for primacy effects instead \citep[e.g.,][]{Camilleri2011}. 

Recently, researchers have explored whether the D-E gap generalizes to valuations. \cite{Golan2014} showed that in both a description- \textit{and} an experience-based task, participants on average valued gambles as if they overweighted rare events. Furthermore, people's valuations indicated stronger overweighting from description than from experience. Crucially, \cite{Golan2014} reported that people drew larger samples to base their valuations on than they typically do to inform their choices. This difference is central, since the D--E--gap in choice is often explained as resulting from small samples. Although sample size and under-sampling are arguably the most discussed reason for the D--E gap, we could not find a systematic analysis of how preference evolves as a function of sample size. 
Therefore, in this paper, we aim to study the influence of sample size on the way preferences are formed from experience and thereby its role in the description--experience gap
%Janine: changed subsequent sentences: Therefore, in this paper, we aim to study the role of sample size in preference formation from experience.
% Gilles: Maybe adjust this sentence, so that the full study follows logically:
% "Therefore, in this paper, we aim to study the influence of sample size on the way preferences are formed from experience and thereby its role in the description--experience gap".


\section{How does sample size influence preference from experience}

% Gilles: deleted this line, too much redundant with the former: The goal of the present article is to examine the impact of sample size on people's preferences from experience. 
Knowing how people treat the amount of information that different sample sizes offer will help to understand why people rely on small samples in the first place and why preferences differ when based on experience rather than description.
%Gilles: \textnf{Gilles: maybe add ``and why preferences differ when based on experience rather than description''}  DONE!!!!
Below, we outline two elementary different views on how sample size could effect preferences.

\subsection{Belief in the law of small numbers}
% Gilles: I deleted the following, because there is little info. Want to get to the point more quickly and stress the structure of the paragraphs below. ``Cognitive psychology has detected many ways of how people receive uncertainty and form beliefs about the probability of uncertain events.''
% The first view on the role of sample size builds on the work of \cite{Tversky1971} who showed-- 




\section{Choices, valuations, and the D-E-gap.}

People's preferences are , thus assessing their 

The results are mixed. While many studies have reported a gap \citep{Hau2008,Hau2010, Ungemach2009} others have not \citep{Camilleri2009,Camilleri2011a,Rakow2008,Kellen2016, Gloeckner2016}. Although it is an interesting question, whether people's preferences systematically differ between description and experience, here we will mainly focus on how people's preferences evolve over the course of sampling. In particular, we explore how the amount of information about a gamble, i.e., the size of the sample drawn, influences people's preferences from experience.


Given this undersampling, it appears obvious that people behave as if they underweight rare events when deciding from experience.

Therefore, 

While much attention has been dedicated to understanding how sample size impact the D--E gap, less attention has been dedicated to understanding how preferences evolve over the course of sampling. 



Traditionally, scholars have studied how people develop preferences in descriptive paradigms. In such descriptive paradigms, gambles are illustrated with a numerical, graphical, or text-based description of their outcome distributions. In this paper, we study how people evaluate risky gambles from experience. In particular, we study how the amount of information shapes people's preferences. 


So far, researchers have mainly focused on studying two streams of research. The first investigates how preferences from experience differ from preferences from description. The second stream studies the cognitive mechanisms that are involved when people develop preferences from experience. 
\subsection{How do preferences differ between description and experience?}





