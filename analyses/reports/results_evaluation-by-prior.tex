\documentclass[a4paper, man, floatsintext]{apa6}
\usepackage{lmodern}
\usepackage{amssymb,amsmath}
\usepackage{ifxetex,ifluatex}
\usepackage{fixltx2e} % provides \textsubscript
\ifnum 0\ifxetex 1\fi\ifluatex 1\fi=0 % if pdftex
  \usepackage[T1]{fontenc}
  \usepackage[utf8]{inputenc}
\else % if luatex or xelatex
  \ifxetex
    \usepackage{mathspec}
  \else
    \usepackage{fontspec}
  \fi
  \defaultfontfeatures{Ligatures=TeX,Scale=MatchLowercase}
\fi
% use upquote if available, for straight quotes in verbatim environments
\IfFileExists{upquote.sty}{\usepackage{upquote}}{}
% use microtype if available
\IfFileExists{microtype.sty}{%
\usepackage{microtype}
\UseMicrotypeSet[protrusion]{basicmath} % disable protrusion for tt fonts
}{}
\usepackage{hyperref}
\hypersetup{unicode=true,
            pdfauthor={Jana B. Jarecki},
            pdfborder={0 0 0},
            breaklinks=true}
\urlstyle{same}  % don't use monospace font for urls
\usepackage{graphicx,grffile}
\makeatletter
\def\maxwidth{\ifdim\Gin@nat@width>\linewidth\linewidth\else\Gin@nat@width\fi}
\def\maxheight{\ifdim\Gin@nat@height>\textheight\textheight\else\Gin@nat@height\fi}
\makeatother
% Scale images if necessary, so that they will not overflow the page
% margins by default, and it is still possible to overwrite the defaults
% using explicit options in \includegraphics[width, height, ...]{}
\setkeys{Gin}{width=\maxwidth,height=\maxheight,keepaspectratio}
\IfFileExists{parskip.sty}{%
\usepackage{parskip}
}{% else
\setlength{\parindent}{0pt}
\setlength{\parskip}{6pt plus 2pt minus 1pt}
}
\setlength{\emergencystretch}{3em}  % prevent overfull lines
\providecommand{\tightlist}{%
  \setlength{\itemsep}{0pt}\setlength{\parskip}{0pt}}
\setcounter{secnumdepth}{0}
% Redefines (sub)paragraphs to behave more like sections
\ifx\paragraph\undefined\else
\let\oldparagraph\paragraph
\renewcommand{\paragraph}[1]{\oldparagraph{#1}\mbox{}}
\fi
\ifx\subparagraph\undefined\else
\let\oldsubparagraph\subparagraph
\renewcommand{\subparagraph}[1]{\oldsubparagraph{#1}\mbox{}}
\fi

%%% Use protect on footnotes to avoid problems with footnotes in titles
\let\rmarkdownfootnote\footnote%
\def\footnote{\protect\rmarkdownfootnote}

%%% Change title format to be more compact
\usepackage{titling}

% Create subtitle command for use in maketitle
\providecommand{\subtitle}[1]{
  \posttitle{
    \begin{center}\large#1\end{center}
    }
}

\setlength{\droptitle}{-2em}

  \title{}
    \pretitle{\vspace{\droptitle}}
  \posttitle{}
    \author{Jana B. Jarecki}
    \preauthor{\centering\large\emph}
  \postauthor{\par}
      \predate{\centering\large\emph}
  \postdate{\par}
    \date{02 Dezember, 2019}

\usepackage{natbib} \usepackage{threeparttable} \usepackage{booktabs}
\shorttitle{test} \usepackage{setspace}
\AtBeginEnvironment{tabular}{\singlespacing} \usepackage{times}
\usepackage{changes} \definechangesauthor[name={JJ}, color=orange]{jj}
\usepackage{upgreek} \AtBeginDocument{\let\maketitle\relax}

\begin{document}

To test these hypotheses, participants were classified as
relative-frequency learners (\(n=25\)), Bayesians with gain priors
\(\theta_G > 1\) (\(n=14\)), and Bayesians with zero-outcome priors
\(\theta_G \leq 1\) (\(n=32\)), based on the individually best-fitting
cognitive model. Participants were excluded if neither the BVU nor the
RF model described them qualitatively or quantitatively (\(n=9\)).

\added[id=jj]{Figure \ref{fig:fig6}a shows that, as predicted, among the Bayesian participants, those with zero-outcome priors only increased their evaluations of p-bets, those with gain priors decreased their evaluations of \$-bets, and that RF-type participants remained relatively stable across sample sizes. Statistical analyses by means of a Bayesian mixed-effects model}\footnote{Regressing the (normalized, within-person z-standardized) evaluations on the predictors sample size category (coded as 1,2,3,4), gamble type, and learner class (BVU-gain-prior, BVU-loss-prior, RF) and interactions of the predictors, with a by-participant random intercept; categorical predictors were effects-coded to facilitate interpretation of interactions.}
\added[id=jj]{
supports a model that includes interactions between sample size and learner classification and between gamble type and learner classification ($M_1$) over a model without the first interaction term ($BF\textsubscript{10} = 12$) and over a model without a the second interaction term  ($BF\textsubscript{20} > 1000$). With weakly informative priors, the modal posterior regression coefficient estimates for sample size for Bayesian learners with zero-outcome priors given p-bets was positive, $b_\textsubscript{BVU,zero,p-bet}$ $=0.14$ (89\% HDI $0.11, 0.18$), and for Bayesians with gain priors given \$-bets the estimate was negative, $b_\textsubscript{BVU,gain,\$-bet}$ $=-0.01$ (89\% HDI $-0.09, 0.00$), as hypothesized. Contrary to the hypotheses, however, for RF-type participants the estimated regression coefficients were not zero but \$-bets $b_\textsubscript{RF,\$-bet}$ $=0.05$ (89\% HDI $0.04, 0.12$) and for p-bets $b_\textsubscript{RF,p-bet}$ $=-0.00$ (89\% HDI $-0.08, -0.00$).}

\begin{figure}[htb]

{\centering \includegraphics[width=.9\linewidth]{../figures/fig6-1} 

}

\caption{Mean evaluations by gamble type and best-fitting cognitive model and prior beliefs of the BVU model. \textit{BVU}$=$Bayesian value updating model, \textit{RF}$=$ Relative frequency model. Error bars indicate standard errors. Sample size categories see Table \ref{table:Lotteries}. Evaluations are scaled to 0-1 and z-standardized at the individual level.}\label{fig:fig6}
\end{figure}

\textit{Predictions about confidence.}
\added[id=jj]{Regarding uncertainty about beliefs, the Bayesian model predicts that the uncertainty about beliefs reduces with more evidence. Thus in Bayesian learners, confidence should increase with growing sample size. The relative frequency model makes no predictions about confidence. To test this, we classified participants into Bayesian and relative-frequency learners based on the best-fitting model. Mean confidence ratings of Bayesian-type learners, z-standardized within participants, increased only slightly from the extra small sample size ($M=-0.07, SD=1.00$), to the small sample size ($M=0.01, SD=0.91$), medium sample size ($M=-0.01, SD=1.00$), and large sample size ($M=0.07, SD=1.00$, Figure \ref{fig:fig6}b), but statistical analyses showed that a linear regression model\footnote{Regressing the (within-person z-standardized) confidence ratings on the predictors sample size category (coded as 1,2,3,4), gamble type, and winning model (BVU, RF), and interactions of the predictors, with a by-participant random intercept; categorical predictors were effects-coded to facilitate interpretation of interactions.} ($M_1$) that includes the sample size as predictor fitted the data worse than an intercept-only model ($BF_{01}> 1000$).}


\end{document}
