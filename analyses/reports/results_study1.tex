\documentclass[a4paper, man, floatsintext]{apa6}
\usepackage{lmodern}
\usepackage{amssymb,amsmath}
\usepackage{ifxetex,ifluatex}
\usepackage{fixltx2e} % provides \textsubscript
\ifnum 0\ifxetex 1\fi\ifluatex 1\fi=0 % if pdftex
  \usepackage[T1]{fontenc}
  \usepackage[utf8]{inputenc}
\else % if luatex or xelatex
  \ifxetex
    \usepackage{mathspec}
  \else
    \usepackage{fontspec}
  \fi
  \defaultfontfeatures{Ligatures=TeX,Scale=MatchLowercase}
\fi
% use upquote if available, for straight quotes in verbatim environments
\IfFileExists{upquote.sty}{\usepackage{upquote}}{}
% use microtype if available
\IfFileExists{microtype.sty}{%
\usepackage{microtype}
\UseMicrotypeSet[protrusion]{basicmath} % disable protrusion for tt fonts
}{}
\usepackage{hyperref}
\hypersetup{unicode=true,
            pdfauthor={Jana B. Jarecki},
            pdfborder={0 0 0},
            breaklinks=true}
\urlstyle{same}  % don't use monospace font for urls
\usepackage{graphicx,grffile}
\makeatletter
\def\maxwidth{\ifdim\Gin@nat@width>\linewidth\linewidth\else\Gin@nat@width\fi}
\def\maxheight{\ifdim\Gin@nat@height>\textheight\textheight\else\Gin@nat@height\fi}
\makeatother
% Scale images if necessary, so that they will not overflow the page
% margins by default, and it is still possible to overwrite the defaults
% using explicit options in \includegraphics[width, height, ...]{}
\setkeys{Gin}{width=\maxwidth,height=\maxheight,keepaspectratio}
\IfFileExists{parskip.sty}{%
\usepackage{parskip}
}{% else
\setlength{\parindent}{0pt}
\setlength{\parskip}{6pt plus 2pt minus 1pt}
}
\setlength{\emergencystretch}{3em}  % prevent overfull lines
\providecommand{\tightlist}{%
  \setlength{\itemsep}{0pt}\setlength{\parskip}{0pt}}
\setcounter{secnumdepth}{0}
% Redefines (sub)paragraphs to behave more like sections
\ifx\paragraph\undefined\else
\let\oldparagraph\paragraph
\renewcommand{\paragraph}[1]{\oldparagraph{#1}\mbox{}}
\fi
\ifx\subparagraph\undefined\else
\let\oldsubparagraph\subparagraph
\renewcommand{\subparagraph}[1]{\oldsubparagraph{#1}\mbox{}}
\fi

%%% Use protect on footnotes to avoid problems with footnotes in titles
\let\rmarkdownfootnote\footnote%
\def\footnote{\protect\rmarkdownfootnote}

%%% Change title format to be more compact
\usepackage{titling}

% Create subtitle command for use in maketitle
\providecommand{\subtitle}[1]{
  \posttitle{
    \begin{center}\large#1\end{center}
    }
}

\setlength{\droptitle}{-2em}

  \title{}
    \pretitle{\vspace{\droptitle}}
  \posttitle{}
    \author{Jana B. Jarecki}
    \preauthor{\centering\large\emph}
  \postauthor{\par}
      \predate{\centering\large\emph}
  \postdate{\par}
    \date{06 Januar, 2020}

\usepackage{natbib} \usepackage{threeparttable} \usepackage{booktabs}
\shorttitle{test} \usepackage{setspace}
\AtBeginEnvironment{tabular}{\singlespacing} \usepackage{times}
\usepackage{changes} \definechangesauthor[name={JJ}, color=orange]{jj}
\usepackage{upgreek} \AtBeginDocument{\let\maketitle\relax}

\begin{document}

\subsubsection{Evaluations of gambles and sample sizes}

Table \ref{tab:means_study1} summarizes the evaluations of the gambles
by format (description vs.~experience) and sample size (xs, s, m, l).
For instance, across sample sizes xs, s, m, and l, the participants
evaluated gamble 4 with an average of \(2.80, 2.91, 2.80, 2.95, 3.08\)
(respectively). Across gambles, the evaluations were not influenced by
sample sizes; a statistical model comparison among analyses of variance
(ANOVA) confirmed this, an ANOVA of the valuations with the predictors
gamble type and gamble expected value (Model 0) outperformed models with
the additional predictors sample size (\(BF_{01} = 420\)) and a
sample-size x gamble-type interaction (\(BF_{02} > 1000\), all models
had by-participant random effects). Although the \$-bets and p-bets had
the same expected values, the \$-bet gambles were evaluated higher
(\(M=5.78, SD=3.94\)) than the p-bet gambles (\(M=2.74, SD=1.09\)),
\(BF > 1000\) in favor of an ANOVA that predicts valuations as a
function of gamble type over an ANOVA without the predictor gamble type
(both models include by-participant and by-expected-value random
effects).

The valuations in the experience condition were not driven by primacy or
recency: neither the outcomes that participants sampled in the first
half of the sampling phase nor those sampled during the second half of
the sampling phase predict the observed valuations. A linear model of
the evaluations with the predictor gamble type
(\(\mathrm{M}\textsubscript{0}\)) outperformed a model with the
additional predictor mean of the first half of samples
(\(BF\textsubscript{01} = 16.6\)) and a model with the additional
predictor mean of the second half of samples
(\(BF\textsubscript{02} = 6.7\)).

\textit{Description versus experience.} Participants' valuations from
description and experience differed for most of the gambles and sample
sizes (Table \ref{tab:means_study1}, rightmost column). The \$-bets were
evaluated higher based on experience (\(M=5.78, SD=3.94\)) compared to
description (\(M=4.93, SD=3.45\)). By contrast p-bets were evaluated
lower on the basis of experience (\(M=2.74, SD=1.09\)) compared to
description (\(M=2.86, SD=0.98\)), \(BF > 1000\) in favor of an ANOVA
including a gamble-type x condition interaction over a model with only
the main effects. Thus, we found a D--E gap that differs from the
classic D--E gap observed in choice paradigms. In Study 1, participants
valued gambles as if they overweighted rare events from description
\textit{and} experience. This effect was even stronger when participants
made valuations from experience.

\begin{table}[tbp]

\begin{center}
\begin{threeparttable}

\caption{\label{tab:means_study1}Valuations of Gambles in Study 1}

\begin{tabular}{lccccrr}
\toprule
Condition & Sample size category & Sample size & \textit{Med} & \textit{M} & D--E & D--E:$BF\textsubscript{10}$\\
\midrule
Gamble ID 1 (\$-bet) &  &  &  &  &  & \\
\ \ \ E & xs & 5 & 5.00 & 5.16 & -0.56 & 4\\
\ \ \ E & s & 10 & 4.55 & 5.30 & -0.70 & 61\\
\ \ \ E & m & 15 & 5.00 & 5.34 & -0.74 & 25\\
\ \ \ E & l & 30 & 5.00 & 5.29 & -0.69 & 146\\
\ \ \ D & -- & -- & 4.00 & 4.60 & -- & --\\
Gamble ID 2 (\$-bet) &  &  &  &  &  & \\
\ \ \ E & xs & 6 & 4.00 & 4.33 & -0.71 & 133\\
\ \ \ E & s & 12 & 4.00 & 4.31 & -0.69 & 631\\
\ \ \ E & m & 18 & 4.00 & 4.04 & -0.43 & 4\\
\ \ \ E & l & 36 & 4.00 & 3.99 & -0.37 & 2\\
\ \ \ D & -- & -- & 3.00 & 3.61 & -- & --\\
Gamble ID 3 (\$-bet) &  &  &  &  &  & \\
\ \ \ E & xs & 7 & 6.00 & 7.56 & -0.99 & 4\\
\ \ \ E & s & 14 & 6.70 & 8.40 & -1.83 & 905\\
\ \ \ E & m & 21 & 6.20 & 7.92 & -1.35 & 138\\
\ \ \ E & l & 42 & 6.00 & 7.68 & -1.11 & 16\\
\ \ \ D & -- & -- & 5.00 & 6.57 & -- & --\\
Gamble ID 4 (p-bet) &  &  &  &  &  & \\
\ \ \ E & xs & 5 & 3.00 & 2.80 & 0.28 & >1000\\
\ \ \ E & s & 10 & 3.20 & 2.91 & 0.16 & 3\\
\ \ \ E & m & 15 & 3.00 & 2.80 & 0.28 & 59\\
\ \ \ E & l & 30 & 3.00 & 2.95 & 0.13 & 1\\
\ \ \ D & -- & -- & 3.20 & 3.08 & -- & --\\
Gamble ID 5 (p-bet) &  &  &  &  &  & \\
\ \ \ E & xs & 6 & 2.00 & 1.77 & 0.10 & 8\\
\ \ \ E & s & 12 & 2.00 & 1.75 & 0.12 & 14\\
\ \ \ E & m & 18 & 2.00 & 1.73 & 0.14 & 27\\
\ \ \ E & l & 36 & 2.00 & 1.81 & 0.06 & 1\\
\ \ \ D & -- & -- & 2.00 & 1.87 & -- & --\\
Gamble ID 6 (p-bet) &  &  &  &  &  & \\
\ \ \ E & xs & 7 & 4.00 & 3.46 & 0.18 & 3\\
\ \ \ E & s & 14 & 4.00 & 3.55 & 0.09 & 0\\
\ \ \ E & m & 21 & 4.00 & 3.58 & 0.06 & 0\\
\ \ \ E & l & 42 & 4.00 & 3.70 & -0.06 & 0\\
\ \ \ D & -- & -- & 4.00 & 3.64 & -- & --\\
\bottomrule
\addlinespace
\end{tabular}

\begin{tablenotes}[para]
\normalsize{\textit{Note.} \textit{M} = mean, \textit{Med} = median, D--E = difference between mean description-based valuations and experience-based valuations, $BF\textsubscript{10}$ = Bayes Factor quantifying the evidence for a linear model $\mathrm{M}\textsubscript{1}$ predicting that valuations differ between description and experience over a linear model $\mathrm{M}\textsubscript{0}$ predicting no such differences; both models models contain a by-participant random effect. Gambles IDs 1, 2, and 3 are \$-bets; Gamble IDs 4, 5, and 6 are p-bets.}
\end{tablenotes}

\end{threeparttable}
\end{center}

\end{table}

At the aggregate level, sample size seems to not influence the
evaluations of gambles in our decision from experience paradigm, but the
subsequent cognitive modeling analyses will show a more nuanced picture.

\subsubsection{Cognitive modeling}

By modeling participant's valuations with the two models we proposed, it
is possible to examine the effect of sample size on value judgments. We
tested how good the \added[id=jj]{relative frequency} (RF) model and the
\added[id=jj]{Bayesian value updating} (BVU) model represent
participants' value judgments of gambles. To check the psychological
plausibility of both models we also checked if both models outperform a
baseline model, which predicts a constant evaluation equal to the mean
individual evaluation.

\textit{Modeling Procedure.} To estimate the models, the observed and
predicted evaluations were normalized to a common range between 0 and 1
(by division through the gain of the presented gamble). Maximum
likelihood was used to estimate the parameters of the models at the
participant level, assuming that observations follow a truncated normal
distribution around the model predictions (truncation between 0 and 1)
with a constant standard deviation (\(\sigma\)), that was estimated by
participant (\(0 < \sigma \leq 1\)).
\added[id=jj]{Therefore, the relative frequency model had two free parameters, namely $\sigma$ and the power utility exponent $\tau$ ($0 \leq \tau \leq 20$). The Bayesian value updating model had four free parameters, the prior belief parameter for gain outcomes $\alpha_0$ ($0 \leq \alpha_0 \leq 2$), the learning rate $\delta$ ($0 \leq \delta \leq 10$), $\tau$ and $\sigma$, with the prior on zero outcomes constrained to $\beta_0 = 2-\alpha_0$. The baseline model had two free parameters, the mean evaluation $\mu$ and $\sigma$. We estimated the parameters using an augmented Lagrange multiplier method \citep[Rsolnp package, version 1.16]{Ghalanos2015}. Models were compared based on evidence strength and BIC weights from the Bayesian information criterion (BIC) \cite[evidence in favor of a model compared to the individually best-fitting model][]{Kass1995, Lewandowsky2011}. Higher weights indicate stronger evidence for a model.}

We will first outline the quantitative model fit, followed by the
qualitative model fit, and lastly analyze the effects of sample size
given the cognitive strategies.

\textit{Quantitative Model Fit.}
\added[id=jj]{The Bayesian value updating model described the majority of the participants best (30 of 40; 75\%). The relative frequency model described 9 participants best (22\%); the baseline model described 1 participants best. Figure \ref{fig:fig2} shows the evidence strength for the models by participant. The models' mean Bayesian information criterion across all participants equaled BIC\textsubscript{BVU}$= -124$, BIC\textsubscript{RF}$= -110$, and BIC\textsubscript{BASE}$= -17$ (lower values indicate better fit).}

\begin{figure}[htb]

{\centering \includegraphics[width=.9\linewidth]{../figures/fig2-1} 

}

\caption{Study 1: Evidence for the models for individual participants. \textit{RF}$=$ relative frequency model, \textit{BVU}$=$ Bayesian value updating model, \textit{BASE}$=$ Baseline model.}\label{fig:fig2}
\end{figure}

\added[id=jj]{
The estimated parameters of the best-fitting models are shown in Table \ref{tab:study1_parameter}. The power utility exponent ($\tau$) did not differ substantially for the participants that were best described by the Bayesian value updating (BVU) model ($M_{\tau}= 1.49$) and those best described by a relative frequency model ($M_{\tau}=1.61$), $\Delta$ $M = -0.07$ 95\% HDI $[-0.92$, $0.82]$, $\mathrm{BF}_{\textrm{01}} = 2.76$. The prior beliefs about the probability of gains was $0.46$ (model parameters $\alpha_0 = 0.92$, $\beta_0 = 1.08$) for the participants best described by the BVU model. The learning rate of the mean participant that was best described by the BVU model was $M_{\delta}=1.36$ (a model with values $>$ 1 revises its priors faster than optimal Bayesian, 1 is optimal Bayesian, $<$ 1 is conservative compared to optimal Bayesian). The liberal learning rate is unusual given that previous work has found conservative learning \citep{Edwards1967,Tauber2017}. This fast learning in our task can be explained by that participants repeatedly sampled from the same set of gambles.
}

\begin{table}[tbp]

\begin{center}
\begin{threeparttable}

\caption{\label{tab:study1_parameter}Study 1: Parameter Estimates of Winning Models, \textit{M (SD)}}

\begin{tabular}{lccccc}
\toprule
Winning Model & $\alpha$ & $\delta$ & $\theta_G$ & $\mu$ & $\sigma$\\
\midrule
BVU (\textit{n}$=$30) & 1.49 (1.47) & 1.36 (2.12) & 0.92 (0.66) & -- & 0.13 (0.07)\\
RF (\textit{n}$=$9) & 1.61 (0.58) & -- & -- & -- & 0.12 (0.02)\\
BASE (\textit{n}$=$1) & -- & -- & -- & 0.47 (NA) & 0.30 (NA)\\
\bottomrule
\addlinespace
\end{tabular}

\begin{tablenotes}[para]
\normalsize{\textit{Note.} \textit{BVU}$=$ Bayesian value updating model, \textit{RF}$=$ relative frequency model, \textit{BASE}$=$baseline model. Parameters denote: $\tau =$ power utility exponent, $\alpha_0$ gain prior, $\mu=$ mean evaluation, $\sigma$ standard deviation.}
\end{tablenotes}

\end{threeparttable}
\end{center}

\end{table}

\textit{Qualitative Model Fit.} The qualitative fit between the models
and the data is shown in Figure \ref{fig:fig5}, which plots the
predictions of the best-fitting models against the observed evaluations
by participant. The models generally describe the data well
\added[id=jj]{($M r\textsubscript{pred,obs} = 0.71$), except in four cases, where even the winning model fails to resemble the data qualitatively (participants s05, s19, s24, s38, with $r\textsubscript{pred,obs} < 0.40$).}\footnote{As robustness check we repeated the model comparison with subjective probability weighting, using Prelec’s \citeyear{Prelec1998} single parameter weighting function. This weighting function incorporates non-linearities in the perception of probabilities. However, the quantitative results of the probability weighting model and the utility model, we favored a utility model without probability weighting.}

\begin{figure}[htb]

{\centering \includegraphics[width=\textwidth]{../figures/fig5-1} 

}

\caption{Study 1: Predicted evaluations from the best-fitting models plotted against the observed evaluations (by participant). \textit{BVU}$=$ Bayesian value updating model, \textit{RF}$=$ relative frequency model, \textit{BASE}$=$baseline model.}\label{fig:fig5}
\end{figure}

\added[id=jj]{
The cognitive modeling results thus show that most participants were best described by a Bayesian strategy, and a minority by a relative-frequency strategy. This strategy heterogeneity helps understanding the behavioral null finding---that sample size seemed to have no effect on valuations---that were observed at the aggregate level (Table \ref{tab:means_study1}). The statistical aggregate analysis fails to take the individual differences in learning strategies into account, while participants are best described by a mixture of strategies. Moreover, the aggregate statistical analysis without the cognitive modeling also fails to account for differences in the prior beliefs about gain probabilities.
}


\end{document}
