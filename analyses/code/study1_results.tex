\documentclass[a4paper, man, floatsintext]{apa6}
\usepackage{lmodern}
\usepackage{amssymb,amsmath}
\usepackage{ifxetex,ifluatex}
\usepackage{fixltx2e} % provides \textsubscript
\ifnum 0\ifxetex 1\fi\ifluatex 1\fi=0 % if pdftex
  \usepackage[T1]{fontenc}
  \usepackage[utf8]{inputenc}
\else % if luatex or xelatex
  \ifxetex
    \usepackage{mathspec}
  \else
    \usepackage{fontspec}
  \fi
  \defaultfontfeatures{Ligatures=TeX,Scale=MatchLowercase}
\fi
% use upquote if available, for straight quotes in verbatim environments
\IfFileExists{upquote.sty}{\usepackage{upquote}}{}
% use microtype if available
\IfFileExists{microtype.sty}{%
\usepackage{microtype}
\UseMicrotypeSet[protrusion]{basicmath} % disable protrusion for tt fonts
}{}
\usepackage{hyperref}
\hypersetup{unicode=true,
            pdfauthor={Jana B. Jarecki},
            pdfborder={0 0 0},
            breaklinks=true}
\urlstyle{same}  % don't use monospace font for urls
\usepackage{graphicx,grffile}
\makeatletter
\def\maxwidth{\ifdim\Gin@nat@width>\linewidth\linewidth\else\Gin@nat@width\fi}
\def\maxheight{\ifdim\Gin@nat@height>\textheight\textheight\else\Gin@nat@height\fi}
\makeatother
% Scale images if necessary, so that they will not overflow the page
% margins by default, and it is still possible to overwrite the defaults
% using explicit options in \includegraphics[width, height, ...]{}
\setkeys{Gin}{width=\maxwidth,height=\maxheight,keepaspectratio}
\IfFileExists{parskip.sty}{%
\usepackage{parskip}
}{% else
\setlength{\parindent}{0pt}
\setlength{\parskip}{6pt plus 2pt minus 1pt}
}
\setlength{\emergencystretch}{3em}  % prevent overfull lines
\providecommand{\tightlist}{%
  \setlength{\itemsep}{0pt}\setlength{\parskip}{0pt}}
\setcounter{secnumdepth}{0}
% Redefines (sub)paragraphs to behave more like sections
\ifx\paragraph\undefined\else
\let\oldparagraph\paragraph
\renewcommand{\paragraph}[1]{\oldparagraph{#1}\mbox{}}
\fi
\ifx\subparagraph\undefined\else
\let\oldsubparagraph\subparagraph
\renewcommand{\subparagraph}[1]{\oldsubparagraph{#1}\mbox{}}
\fi

%%% Use protect on footnotes to avoid problems with footnotes in titles
\let\rmarkdownfootnote\footnote%
\def\footnote{\protect\rmarkdownfootnote}

%%% Change title format to be more compact
\usepackage{titling}

% Create subtitle command for use in maketitle
\providecommand{\subtitle}[1]{
  \posttitle{
    \begin{center}\large#1\end{center}
    }
}

\setlength{\droptitle}{-2em}

  \title{}
    \pretitle{\vspace{\droptitle}}
  \posttitle{}
    \author{Jana B. Jarecki}
    \preauthor{\centering\large\emph}
  \postauthor{\par}
      \predate{\centering\large\emph}
  \postdate{\par}
    \date{07 November, 2019}

\usepackage{natbib} \usepackage{threeparttable} \usepackage{booktabs}
\shorttitle{test} \usepackage{setspace}
\AtBeginEnvironment{tabular}{\singlespacing} \usepackage{times}
\usepackage{changes} \definechangesauthor[name={JJ}, color=orange]{jj}
\usepackage{upgreek} \AtBeginDocument{\let\maketitle\relax}

\begin{document}

\subsubsection{Evaluations of gambles from different sample sizes}

Table \ref{tab:means_study1} summarizes participants' evaluations of the
gambles in the experience and description condition. For instance, in
the experience condition, across sample sizes xs, s, m, and l, people
assigned a value of \(2.80, 2.91, 2.80, 2.95, 3.08\) (respectively).
Overall, the different sample sizes had no systematic influence on
participants' evaluations, which was confirmed by comparing analyses of
variance models (ANOVA), showing that evaluations are best predicted by
a model that does not include sample size as predictor but a combination
of gamble type and expected value with a by-participant random effect
(Model 1), which outperformed a model including sample size
(\(BF_{10}= 417\)) and a model including a sample-size-gamble-type
interaction (\(BF_{10} > 1000\)). Although this suggests that sample
size does not reliably influence the evaluations of gambles in a
decision from experience paradigm, the cognitive modeling analyses will
show a more nuanced picture.

\textbackslash{}begin\{table\}{[}tbp{]}

\textbackslash{}begin\{center\} \textbackslash{}begin\{threeparttable\}

\caption{\label{tab:means_study1}Valuations of Gambles in Study 1}

\textbackslash{}begin\{tabular\}\{lccccrr\} \toprule Condition \& Sample
size category \& Sample size \& \textit{Med} \& \textit{M} \& D--E \&
D--E:\(BF\textsubscript{10}\)\textbackslash{} \midrule Gamble ID 1
(\(-bet) & & & & & & \\ \ \ \ E & xs & 5 & 5.00 & 5.16 & -0.56 & 5\\ \ \ \ E & s & 10 & 4.55 & 5.30 & -0.70 & 63\\ \ \ \ E & m & 15 & 5.00 & 5.34 & -0.74 & 24\\ \ \ \ E & l & 30 & 5.00 & 5.29 & -0.69 & 142\\ \ \ \ D & -- & -- & 4.00 & 4.60 & -- & --\\ Gamble ID 2 (\)-bet)
\& \& \& \& \& \& \textbackslash{} ~~~E \& xs \& 6 \& 4.00 \& 4.33 \&
-0.71 \& 134\textbackslash{} ~~~E \& s \& 12 \& 4.00 \& 4.31 \& -0.69 \&
679\textbackslash{} ~~~E \& m \& 18 \& 4.00 \& 4.04 \& -0.43 \&
4\textbackslash{} ~~~E \& l \& 36 \& 4.00 \& 3.99 \& -0.37 \&
2\textbackslash{} ~~~D \& -- \& -- \& 3.00 \& 3.61 \& -- \&
--\textbackslash{} Gamble ID 3 (\$-bet) \& \& \& \& \& \&
\textbackslash{} ~~~E \& xs \& 7 \& 6.00 \& 7.56 \& -0.99 \&
4\textbackslash{} ~~~E \& s \& 14 \& 6.70 \& 8.40 \& -1.83 \&
879\textbackslash{} ~~~E \& m \& 21 \& 6.20 \& 7.92 \& -1.35 \&
144\textbackslash{} ~~~E \& l \& 42 \& 6.00 \& 7.68 \& -1.11 \&
16\textbackslash{} ~~~D \& -- \& -- \& 5.00 \& 6.57 \& -- \&
--\textbackslash{} Gamble ID 4 (p-bet) \& \& \& \& \& \&
\textbackslash{} ~~~E \& xs \& 5 \& 3.00 \& 2.80 \& 0.28 \&
\textgreater{}1000\textbackslash{} ~~~E \& s \& 10 \& 3.20 \& 2.91 \&
0.16 \& 3\textbackslash{} ~~~E \& m \& 15 \& 3.00 \& 2.80 \& 0.28 \&
57\textbackslash{} ~~~E \& l \& 30 \& 3.00 \& 2.95 \& 0.13 \&
2\textbackslash{} ~~~D \& -- \& -- \& 3.20 \& 3.08 \& -- \&
--\textbackslash{} Gamble ID 5 (p-bet) \& \& \& \& \& \&
\textbackslash{} ~~~E \& xs \& 6 \& 2.00 \& 1.77 \& 0.10 \&
8\textbackslash{} ~~~E \& s \& 12 \& 2.00 \& 1.75 \& 0.12 \&
15\textbackslash{} ~~~E \& m \& 18 \& 2.00 \& 1.73 \& 0.14 \&
27\textbackslash{} ~~~E \& l \& 36 \& 2.00 \& 1.81 \& 0.06 \&
1\textbackslash{} ~~~D \& -- \& -- \& 2.00 \& 1.87 \& -- \&
--\textbackslash{} Gamble ID 6 (p-bet) \& \& \& \& \& \&
\textbackslash{} ~~~E \& xs \& 7 \& 4.00 \& 3.46 \& 0.18 \&
3\textbackslash{} ~~~E \& s \& 14 \& 4.00 \& 3.55 \& 0.09 \&
0\textbackslash{} ~~~E \& m \& 21 \& 4.00 \& 3.58 \& 0.06 \&
0\textbackslash{} ~~~E \& l \& 42 \& 4.00 \& 3.70 \& -0.06 \&
0\textbackslash{} ~~~D \& -- \& -- \& 4.00 \& 3.64 \& -- \&
--\textbackslash{} \bottomrule \addlinespace
\textbackslash{}end\{tabular\}

\begin{tablenotes}[para]
\normalsize{\textit{Note.} \textit{M} = mean, \textit{Med} = median, D--E = difference between mean description-based valuations and experience-based valuations, $BF\textsubscript{10}$ = Bayes Factor quantifying the evidence for a linear model $\mathrm{M}\textsubscript{1}$ predicting that valuations differ between description and experience over a linear model $\mathrm{M}\textsubscript{0}$ predicting no such differences; both models models contain a by-participant random effect. Gambles IDs 1, 2, and 3 are \$-bets; Gamble IDs 4, 5, and 6 are p-bets.}
\end{tablenotes}

\textbackslash{}end\{threeparttable\} \textbackslash{}end\{center\}

\textbackslash{}end\{table\}

\subsubsection{Cognitive modeling of experience-based evaluations}

We used computational modeling to analyze the role of sample size in
value judgments more closely, comparing the performance of the
\added[id=jj]{relative frequency} (RF) model and the
\added[id=jj]{Bayesian value updating} (BVU) model.
\added[id=jj]{The models were compared to a baseline model, which predicts a constant evaluation equal to the mean individual evaluation (sensible models are expected to outperform this baseline model).}

\textit{Modeling Procedure} The observed and predicted evaluations were
normalized to a common range between 0 and 1 (by division through the
gain of the respective gamble). Maximum likelihood was used to estimate
the free model parameters at the participant level, assuming that
observations follow a truncated normal distribution around the model
predictions (truncated between 0 and 1) with a constant standard
deviation (\(\sigma\)), that was estimated (\(0 < \sigma \leq 1\)).
\added[id=jj]{Therefore, the relative frequency model had two free parameters, the power utility exponent $\alpha$ ($0 \leq \alpha \leq 20$) and $\sigma$. The Bayesian value updating model had four free parameters, the gain prior $\theta_G$ ($0 \leq \theta_G \leq 1$), the learning rate $\delta$ ($0 \leq \delta \leq 10$), $\alpha$ and $\sigma$, with the prior on zero outcomes constrained to $\theta_0=2-\theta_G$. The baseline model had two free parameters, the mean evaluation $\mu$ and $\sigma$. We estimated the parameters using an augmented Lagrange multiplier method \citep[Rsolnp package, version 1.16]{Ghalanos2015}. Models were compared based on evidence strength and BIC weights from the Bayesian information criterion (BIC) \cite[evidence in favor of a model compared to the individually best-fitting model][]{Kass1995, Lewandowsky2011}. Higher weights indicate stronger evidence for a model.}

\added[id=jj]{We will first outline the quantitative model fit, followed by the qualitative model fit, and lastly analyze the effects of sample size given the cognitive strategies.}

\textit{Quantitative Model Fit.} The Bayesian value updating model
described the majority of the participants best (30 of 40; 75\%). The
relative frequency model described 9 participants best (22\%); the
baseline model described only 1 participants best. Figure
\ref{fig:study1_model_weights} shows the evidence strength for the
models by participant. The models' mean Bayesian information criterion
across all participants equaled BIC\textsubscript{BVU}\(= -124\),
BIC\textsubscript{RF}\(= -110\), and BIC\textsubscript{BASE}\(= -17\)
(lower values indicate better fit).

\added[id=jj]{
The estimated parameter of the winning models, which Table \ref{tab:study1_parameter} summarizes, reveal that the power utility exponent ($\alpha$) is almost identical for the participants using a Bayesian value updating strategy ($M_{\alpha}= 1.43$) and those using a relative frequency strategy ($M_{\alpha}=1.60$), $M = -0.13$ 95\% HDI $[-0.69$, $0.40]$, $\mathrm{BF}_{\textrm{01}} = 3.26$. Participants using the Bayesian strategy had, on average, a prior belief that gains occur with 46\% (gain prior $\theta_G = 0.92$; zero-outcome prior $\theta_0 = 1.08$). Also, their estimated learning rate $\delta$ was anti-conservative ($M_{\delta}=1.36$; values $>$ 1 are liberal, 1 is optimal Bayesian, $<$ 1 is conservative learning). Liberal learning was not observed in previous data which found conservative learning \citep{Edwards1967,Tauber2017}, but the liberal learning in our data can be explained by that participants repeatedly sampled from the same set of gambles.
}

\begin{table}[tbp]

\begin{center}
\begin{threeparttable}

\caption{\label{tab:study1_parameter}Parameter Estimates of Winning Models, \textit{M (SD)}}

\begin{tabular}{lccccc}
\toprule
Winning Model & $\alpha$ & $\delta$ & $\theta_G$ & $\mu$ & $\sigma$\\
\midrule
NA (\textit{n}$=$1) & 0.01 (0.00) & 0.06 (NA) & 0.19 (NA) & 0.47 (NA) & 0.49 (0.16)\\
RF (\textit{n}$=$9) & 1.60 (0.57) & 10.00 (0.00) & 1.15 (0.93) & 0.54 (0.04) & 0.33 (0.35)\\
BVU (\textit{n}$=$30) & 1.43 (1.23) & 1.36 (2.12) & 0.92 (0.66) & 0.53 (0.07) & 0.28 (0.28)\\
\bottomrule
\addlinespace
\end{tabular}

\begin{tablenotes}[para]
\normalsize{\textit{Note.} \textit{BVU}$=$ Bayesian value updating model, \textit{RF}$=$ relative frequency model, \textit{BASE}$=$baseline model. Parameters denote: $\alpha=$ power utility exponent, $\theta_G$ gain prior, $\mu=$ mean evaluation, $\sigma$ standard deviation.}
\end{tablenotes}

\end{threeparttable}
\end{center}

\end{table}

\textit{Qualitative Model Fit.}
\added[id=jj]{The qualitative fit between the models and the data is shown in Figure \ref{fig:ind_fits1}, which plots the predictions of the best-fitting models against the observed evaluations. It shows that the models generally describe the data well (mean $r\textsubscript{pred,obs} = 0.54$), except in four cases, where even the winning model fails to resemble the data qualitatively (participants number 05, 07, 13, 15, 19, 24, 26, 30, 34, 38, 40, with $r\textsubscript{pred,obs} < 0.40$). For these cases, for whom the winning model is the Bayesian updating model, the model must be rejected because of qualitative mis-fit.\footnote{As robustness check we repeated the model comparison with subjective probability weighting, using Prelec’s single parameter weighting function. This weighting function incorporates non-linearities in the perception of probabilities. However, the quantitative results of the probability weighting model and the utility model, we favored a utility model without probability weighting.}}

\begin{figure}[htb]

{\centering \includegraphics{../figures/ind_fits1-1} 

}

\caption{Predicted evaluations from the best-fitting models plotted against the observed evaluations (by participant). \textit{BVU}$=$ Bayesian value updating model, \textit{RF}$=$ relative frequency model, \textit{BASE}$=$baseline model.}\label{fig:ind_fits1}
\end{figure}

\added[id=jj]{The cognitive modeling results thus show that most participants were described by a Bayesian strategy, and a minority by a relative-frequency strategy. This strategy heterogeneity helps understanding the behavioral null finding---that sample size seemed to have no effect on valuations---that were observed at the aggregate level (Table \ref{tab:means_study1}). The aggregate analysis fails to take the individual differences in learning strategies into account, while participants are best described by a mixture of strategies. Moreover, the aggregate analysis also fails to account for differences in the prior beliefs about gain probabilities. Depending on the prior belief, the Bayesian value updating (BVU) model predicts either a decrease or an increase in valuations with increasing sample size. The next analysis will focus on these differences.}

\emph{The effect of sample size given cognitive strategies.} Next, we
qualitatively analyzed if sample size differentially affects the
relative-frequency-type and Bayesian-type learners. We expected that
sample size leads to changes in the evaluations of the Bayesian learners
depending on their priors, and that sample size does not affect the
evaluations of the relative-frequency learners.

\added[id=jj]{
The Bayesian model predicts that the evaluations are influenced by both sample size and prior beliefs. Participants with a gain prior---initially believing that gains are more likely than zero-outcomes---should decrease the evaluations of \$-bets as sample sizes increase, because participants learn that gains of \$-bets are less likely than zero-outcomes. By contrast, participants with a zero-outcome prior---initially believing that zero-outcomes are more likely than gains---should increase their evaluations of p-bets as sample size increases, because they learn that gains of p-bets are more likely than zero-outcomes.
}

\textbackslash{}added{[}id=jj{]}\{Based on the cognitive modeling
results, participants were classified into three learner types,
relative-frequency learners, Bayesian learners with gain priors (prior
\(\theta_G > 1\)), and Bayesian learners with loss priors (prior
\(\theta_G \leq 1\)). Figure \ref{fig:qual1} shows how the learner
types' evaluations of p-bets and \$-bets change with increasing sample
size. The relative-frequency learner types evaluated both p-bets and
\$-bets quite unaffected by sample sizes, whereas the Bayesian learner
types changed their evaluations slightly with sample size in the
predicted directions. Statistical analyses by means of a Bayesian
generalized linear
model\footnote{regressing the (normalized) evaluations on the predictors sample size, gamble type (p-bet, \$-bet), and type (BVU-gain-prior, BVU-loss-prior, RF) with a by-participant random intercept; categorical predictors were effects-coded to facilitate interpretation of interactions \citep[for details, see][]{SingmannForthcoming}), however, showed no substantial support that including the learner type as predictor improves goodness of fit, $BF\textsubscript{01}=0.420567677199355, FALSE$, where 0 $=$ the model including learning type and 1 $=$ the model excluding learning type.}

\begin{figure}[htb]

{\centering \includegraphics{../figures/qual1-1} 

}

\caption{Mean evaluation (standardized to 0 - 1) by winning model and prior beliefs of the BVU model. \textit{BVU}$=$Bayesian value updating model, \textit{RF}$=$ Relative frequency model. Error bars indicate standard errors. \textit{\$-bet}: low-probability high-outcome gambles, \textit{p-bet}: high-probability low-outcome gambles. Sample sizes (xs, x, m, l), see Table \ref{tab:Lotteries}. \textit{n=16, 13, 6} denotes the number of participants best-described by the respective models.}\label{fig:qual1}
\end{figure}

\textit{Effect of gamble type and sampling order} The mean and median
evaluations of \$-bets (\(M (SD) = 5.78 (3.94)\)) exceeded the
evaluations of p-bets (\(M = 2.74 (1.09)\)), despite the fact that the
gambles had the same expected value (Gamble IDs 1--3 versus 4--6, Table
\ref{tab:means_study1}), \(BF_{\text{01}}=> 1000\) in favor of an ANOVA
that predicts evaluations from gamble type with a by-participant and
by-expected-value random effect over an ANOVA without the predictor
gamble type.

To test for recency or primacy effects, we compared how well the first
half and the second half of the sample sequence in each trial predicts
valuations. A linear model \(\mathrm{M}\textsubscript{0}\) that predicts
valuations as a function of the random factor gamble ID outperforms a
model \(\mathrm{M}\textsubscript{1}\), which also includes the mean of
the first half of the observed samples as a fixed factor
(\(BF\textsubscript{01} = 16.6\)) and model
\(\mathrm{M}\textsubscript{2}\), which includes the mean of the second
half of observed samples (\(BF\textsubscript{02} = 6.7\)). In summary,
our analysis did not provide evidence for recency or primacy effects.

\textit{Confidence ratings}

Participants' aggregated mean confidence ratings of their valuations
from experience in the extra small, small, medium, and large sample-size
categories were \(4.11\) (\(SD = 1.10\)), \(4.15\) (\(SD = 1.04\)),
\(4.14\) (\(SD = 1.03\)), and \(4.16\) (\(SD = 1.06\)) for xs, s, m, and
l (respectively). Sample size did not influence participants' confidence
systematically: \(\mathrm{M}\textsubscript{0}\), which predicts
confidence rating as a function of a random participant effect, was
strongly preferred over \(\mathrm{M}\textsubscript{1}\), which in
addition includes the sample-size category as a predictor
(\(BF\textsubscript{01} = 737\)).

\added[id=jj]{According to Bayesian value updating higher sample sizes should increase confidence. The relative frequency model predicts no influence of sample size on confidence. We analyzed the confidence ratings by best-fitting model (Bayesian-type learners and relative-frequency-type learners). The confidence of Bayesian learners did not change remarkably across sample sizes ($M=$ for sample sizes , respectively; $SD$s$=$ , respectively). A linear model\footnote{with by-participant random intercept and the predictors effect-coded.} of the confidence ratings, which excluded sample size as predictor, was preferred over a model including sample size ($BF\textsubscript{excl,incl} = 3.01$).
  Similarly, the confidence of relative-frequency-type learners did not show an effect of sample size on confidence ratings ($M = $ for sample sizes , respectively; \textit{SD}s$=$ , respectively; BF\textsubscript{excl,incl}$= 0$)}.

\textit{Description versus experience} The above Table
\ref{tab:means_study1} shows the mean and median valuations from
description and the difference between the mean valuations in the
experience and description conditions. Further, it provides the Bayes
factors, quantifying the evidence in favor of a difference between
valuations from experience and those from description.

Valuations made from description and experience differed for most of the
gambles and sample sizes (see Table \ref{tab:means_study1}, rightmost
column). In particular, participants attached a higher value to
experienced than to described \$-bets (Gambles 1--3) but attached a
higher value to described than to experienced p-bets (Gambles 4--6).
Thus, we found a D--E gap that is the opposite of the classic D--E gap
observed in choice paradigms. In our study, participants valued gambles
as if they overweighted rare events from description \textit{and}
experience. This effect was even stronger when people made valuations
from experience.

We also compared participants' confidence ratings of valuations from
experience in each sample-size category to those from description
(\(M = 4.04\), \(SD = 1.08\)). Separately for each sample-size category,
we compared \(\mathrm{M}\textsubscript{0}\), which predicts confidence
as a function of random participant effects, with
\(\mathrm{M}\textsubscript{1}\), which takes condition as an additional
fixed factor into account. The analyses suggest that participants were
slightly more confident about their ratings from experience than from
description for small (\(BF\textsubscript{10} = 2.5\)), medium
(\(BF\textsubscript{10} = 1.3\)), and large
(\(BF\textsubscript{10} = 4.8\)) sample sizes. For the extra small
sample sizes (\(BF\textsubscript{10} = 0.3\)), confidence judgments did
not differ.


\end{document}
