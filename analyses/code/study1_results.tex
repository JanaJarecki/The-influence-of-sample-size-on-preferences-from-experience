\documentclass[]{standalone}
\usepackage{lmodern}
\usepackage{amssymb,amsmath}
\usepackage{ifxetex,ifluatex}
\usepackage{fixltx2e} % provides \textsubscript
\ifnum 0\ifxetex 1\fi\ifluatex 1\fi=0 % if pdftex
  \usepackage[T1]{fontenc}
  \usepackage[utf8]{inputenc}
\else % if luatex or xelatex
  \ifxetex
    \usepackage{mathspec}
  \else
    \usepackage{fontspec}
  \fi
  \defaultfontfeatures{Ligatures=TeX,Scale=MatchLowercase}
\fi
% use upquote if available, for straight quotes in verbatim environments
\IfFileExists{upquote.sty}{\usepackage{upquote}}{}
% use microtype if available
\IfFileExists{microtype.sty}{%
\usepackage{microtype}
\UseMicrotypeSet[protrusion]{basicmath} % disable protrusion for tt fonts
}{}
\usepackage{hyperref}
\hypersetup{unicode=true,
            pdftitle={Results},
            pdfauthor={Jana B. Jarecki},
            pdfborder={0 0 0},
            breaklinks=true}
\urlstyle{same}  % don't use monospace font for urls
\usepackage{color}
\usepackage{fancyvrb}
\newcommand{\VerbBar}{|}
\newcommand{\VERB}{\Verb[commandchars=\\\{\}]}
\DefineVerbatimEnvironment{Highlighting}{Verbatim}{commandchars=\\\{\}}
% Add ',fontsize=\small' for more characters per line
\usepackage{framed}
\definecolor{shadecolor}{RGB}{248,248,248}
\newenvironment{Shaded}{\begin{snugshade}}{\end{snugshade}}
\newcommand{\AlertTok}[1]{\textcolor[rgb]{0.94,0.16,0.16}{#1}}
\newcommand{\AnnotationTok}[1]{\textcolor[rgb]{0.56,0.35,0.01}{\textbf{\textit{#1}}}}
\newcommand{\AttributeTok}[1]{\textcolor[rgb]{0.77,0.63,0.00}{#1}}
\newcommand{\BaseNTok}[1]{\textcolor[rgb]{0.00,0.00,0.81}{#1}}
\newcommand{\BuiltInTok}[1]{#1}
\newcommand{\CharTok}[1]{\textcolor[rgb]{0.31,0.60,0.02}{#1}}
\newcommand{\CommentTok}[1]{\textcolor[rgb]{0.56,0.35,0.01}{\textit{#1}}}
\newcommand{\CommentVarTok}[1]{\textcolor[rgb]{0.56,0.35,0.01}{\textbf{\textit{#1}}}}
\newcommand{\ConstantTok}[1]{\textcolor[rgb]{0.00,0.00,0.00}{#1}}
\newcommand{\ControlFlowTok}[1]{\textcolor[rgb]{0.13,0.29,0.53}{\textbf{#1}}}
\newcommand{\DataTypeTok}[1]{\textcolor[rgb]{0.13,0.29,0.53}{#1}}
\newcommand{\DecValTok}[1]{\textcolor[rgb]{0.00,0.00,0.81}{#1}}
\newcommand{\DocumentationTok}[1]{\textcolor[rgb]{0.56,0.35,0.01}{\textbf{\textit{#1}}}}
\newcommand{\ErrorTok}[1]{\textcolor[rgb]{0.64,0.00,0.00}{\textbf{#1}}}
\newcommand{\ExtensionTok}[1]{#1}
\newcommand{\FloatTok}[1]{\textcolor[rgb]{0.00,0.00,0.81}{#1}}
\newcommand{\FunctionTok}[1]{\textcolor[rgb]{0.00,0.00,0.00}{#1}}
\newcommand{\ImportTok}[1]{#1}
\newcommand{\InformationTok}[1]{\textcolor[rgb]{0.56,0.35,0.01}{\textbf{\textit{#1}}}}
\newcommand{\KeywordTok}[1]{\textcolor[rgb]{0.13,0.29,0.53}{\textbf{#1}}}
\newcommand{\NormalTok}[1]{#1}
\newcommand{\OperatorTok}[1]{\textcolor[rgb]{0.81,0.36,0.00}{\textbf{#1}}}
\newcommand{\OtherTok}[1]{\textcolor[rgb]{0.56,0.35,0.01}{#1}}
\newcommand{\PreprocessorTok}[1]{\textcolor[rgb]{0.56,0.35,0.01}{\textit{#1}}}
\newcommand{\RegionMarkerTok}[1]{#1}
\newcommand{\SpecialCharTok}[1]{\textcolor[rgb]{0.00,0.00,0.00}{#1}}
\newcommand{\SpecialStringTok}[1]{\textcolor[rgb]{0.31,0.60,0.02}{#1}}
\newcommand{\StringTok}[1]{\textcolor[rgb]{0.31,0.60,0.02}{#1}}
\newcommand{\VariableTok}[1]{\textcolor[rgb]{0.00,0.00,0.00}{#1}}
\newcommand{\VerbatimStringTok}[1]{\textcolor[rgb]{0.31,0.60,0.02}{#1}}
\newcommand{\WarningTok}[1]{\textcolor[rgb]{0.56,0.35,0.01}{\textbf{\textit{#1}}}}
\usepackage{graphicx,grffile}
\makeatletter
\def\maxwidth{\ifdim\Gin@nat@width>\linewidth\linewidth\else\Gin@nat@width\fi}
\def\maxheight{\ifdim\Gin@nat@height>\textheight\textheight\else\Gin@nat@height\fi}
\makeatother
% Scale images if necessary, so that they will not overflow the page
% margins by default, and it is still possible to overwrite the defaults
% using explicit options in \includegraphics[width, height, ...]{}
\setkeys{Gin}{width=\maxwidth,height=\maxheight,keepaspectratio}
\IfFileExists{parskip.sty}{%
\usepackage{parskip}
}{% else
\setlength{\parindent}{0pt}
\setlength{\parskip}{6pt plus 2pt minus 1pt}
}
\setlength{\emergencystretch}{3em}  % prevent overfull lines
\providecommand{\tightlist}{%
  \setlength{\itemsep}{0pt}\setlength{\parskip}{0pt}}
\setcounter{secnumdepth}{0}
% Redefines (sub)paragraphs to behave more like sections
\ifx\paragraph\undefined\else
\let\oldparagraph\paragraph
\renewcommand{\paragraph}[1]{\oldparagraph{#1}\mbox{}}
\fi
\ifx\subparagraph\undefined\else
\let\oldsubparagraph\subparagraph
\renewcommand{\subparagraph}[1]{\oldsubparagraph{#1}\mbox{}}
\fi

%%% Use protect on footnotes to avoid problems with footnotes in titles
\let\rmarkdownfootnote\footnote%
\def\footnote{\protect\rmarkdownfootnote}

%%% Change title format to be more compact
\usepackage{titling}

% Create subtitle command for use in maketitle
\providecommand{\subtitle}[1]{
  \posttitle{
    \begin{center}\large#1\end{center}
    }
}

\setlength{\droptitle}{-2em}

  \title{Results}
    \pretitle{\vspace{\droptitle}\centering\huge}
  \posttitle{\par}
    \author{Jana B. Jarecki}
    \preauthor{\centering\large\emph}
  \postauthor{\par}
      \predate{\centering\large\emph}
  \postdate{\par}
    \date{Aug.~2019}


\begin{document}

For all analyses we used the software R \citep{R2019}. For the linear
model comparisons, we used the BayesFactor package \citep{BayesFactor}.
We based our inferences on the Bayes factor (\(BF\textsubscript{ij}\)),
which quantifies how much more or less likely the data are under Model i
(\(\mathrm{M}\textsubscript{i}\)) than Model j
(\(\mathrm{M}\textsubscript{j}\)). A Bayes factor of 1
(\(BF\textsubscript{ij} = 1\)) means that the data do not discriminate
between the two models \(\mathrm{M}\textsubscript{i}\) and
\(\mathrm{M}\textsubscript{j}\). A Bayes factor of 5 (i.e.,
\(BF\textsubscript{ij} = 5\)) means that the data are 5 times more
likely under \(\mathrm{M}\textsubscript{i}\) than under
\(\mathrm{M}\textsubscript{j}\). A Bayes factor of \(1/5\)
(\(BF\textsubscript{ij} = 1/5\)) means that the data are 5 times more
likely under \(\mathrm{M}\textsubscript{j}\) (note that
\(BF\textsubscript{ji} = 1/BF\textsubscript{ij}\)).

Table \ref{table:meansStudy2} displays the mean and median valuations
from experience, that is, the monetary amount elicited with the BDM
method, separately for each gamble and each sample-size category. The
data suggest that sample size does not affect valuations from experience
consistently. A model comparison supports this indication. Model
\(\mathrm{M}\textsubscript{0}\), which predicts valuations from
experience as a function of the random factor expected value and the
fixed factor gamble type (p-bet vs. \$-bet), is preferred over a model
that takes into account sample size as an additional fixed factor
(\(BF\textsubscript{01} = 409\)) and a model that additionally takes
into account the interaction between sample size and gamble type
(\(BF\textsubscript{01} > 1,000\)).

\begin{table}[H]
\begin{center}
\begin{threeparttable}
\caption{\label{tab:unnamed-chunk-1}Gambles and Sample Sizes Used in Studies 1 and 2\{table:Lotteries\}\}}
\begin{tabular}{llccccccc}
\toprule
 &  &  &  &  & \multicolumn{4}{c}{Sample Size} \\
\cmidrule(r){6-9}
Gamble ID & Type & X & Pr & EV & xs & s & m & l\\
\midrule
1 & \$-bet & 16.00 & 0.20 & 3.20 & 5 & 10 & 15 & 30\\
2 & \$-bet & 12.00 & 0.17 & 2.00 & 6 & 12 & 18 & 36\\
3 & \$-bet & 28.00 & 0.14 & 4.00 & 7 & 14 & 21 & 42\\
4 & p-bet & 4.00 & 0.80 & 3.20 & 5 & 10 & 15 & 30\\
5 & p-bet & 2.40 & 0.83 & 2.00 & 6 & 12 & 18 & 36\\
6 & p-bet & 4.70 & 0.86 & 4.03 & 7 & 14 & 21 & 42\\
\bottomrule
\addlinespace
\end{tabular}
\begin{tablenotes}[para]
\normalsize{\textit{Note.}  extit\{X\} = gain in Swiss Fr.,     extit\{Pr\} = probability of gain,  extit\{EV\} = expected value,   extit\{Sample Size\} = total number of observations in the experience condition categorized as  extit\{xs\} = extra small,  extit\{s\} = small,     extit\{m\} = medium,    extit\{l\} = large. The probability is expressed as the ratio of the relative frequency of the number of gain observations to the number of observations in the smallest sample size category (xs) of this gamble, namely 1/5, 1/6, 1/7, 4/5, 5/6, and 6/7 for gamble IDS 1 through 6, respectively.}
\end{tablenotes}
\end{threeparttable}
\end{center}
\end{table}


\end{document}
